\documentclass[10pt]{elsarticle}

  \usepackage{pgfplots}
\pgfplotsset{compat=newest}
%% the following commands are needed for some matlab2tikz features
\usetikzlibrary{plotmarks}
\usetikzlibrary{arrows.meta}
\usepgfplotslibrary{patchplots}
\usepackage{grffile}
\usepackage{amsmath}
\usepackage{lineno}


%\usepackage{fullpage}
\usepackage[top=1in, bottom=1in, left=0.8in, right=1in]{geometry}
\usepackage{multicol}
\usepackage{caption}
\usepackage{subcaption}
\usepackage{hyperref}
\usepackage{xcolor}
\usepackage{graphicx,psfrag}
\usepackage[pdf]{pstricks}

\definecolor{lightblue}{rgb}{.80,.9,1}
\newcommand{\hl}[1]
    {\par\colorbox{lightblue}{\parbox{\linewidth}{#1}}}

\newcommand{\defn}{\stackrel{\textrm{\scriptsize def}}{=}}

\setlength{\columnsep}{0.1pc}

\title{Serre Equations with Weak Surface Tension}
%\author{Christopher Zoppou -- \texttt{christopher.zoppou@anu.edu.au}, Dimitrios Mitsotakis -- \texttt{dmitsot@gmail.com}, Stephen Roberts -- \texttt{stephen.roberts@anu.edu.au}, Jordan Pitt}

% TIME ON EVERY PAGE AS WELL AS THE FILE NAME
\usepackage{fancyhdr}
\usepackage{currfile}
\usepackage[us,12hr]{datetime} % `us' makes \today behave as usual in TeX/LaTeX
\fancypagestyle{plain}{
\fancyhf{}
\rfoot{\emph{\footnotesize \textcopyright  Serre Notes by C. Zoppou, D. Mitsatakis and S. Roberts.}
 \\ File Name: {\currfilename} \\ Date: {\ddmmyyyydate\today} at \currenttime}
\lfoot{Page \thepage}
\renewcommand{\headrulewidth}{0pt}}
\pagestyle{plain}

\definecolor{mycolor1}{rgb}{0.00000,0.44700,0.74100}%
\definecolor{mycolor2}{rgb}{0.85000,0.32500,0.09800}%
\definecolor{mycolor3}{rgb}{0.92900,0.69400,0.12500}%
\definecolor{mycolor4}{rgb}{0.49400,0.18400,0.55600}%
\definecolor{mycolor5}{rgb}{0.46600,0.67400,0.18800}% 
\definecolor{mycolor6}{rgb}{0.30100,0.74500,0.93300}%
\definecolor{mycolor7}{rgb}{0.63500,0.07800,0.18400}%

\newcommand\T{\rule{0pt}{3ex }}       % Top table strut
\newcommand\B{\rule[-4ex]{0pt}{4ex }} % Bottom table strut

\newcommand\TM{\rule{0pt}{2.8ex }}       % Top matrix strut
\newcommand\BM{\rule[-2ex]{0pt}{2ex }} % Bottom matrix strut

\newcommand{\vecn}[1]{\boldsymbol{#1}}
\DeclareRobustCommand{\solidrule}[1][0.25cm]{\rule[0.5ex]{#1}{1.5pt}}

\DeclareRobustCommand{\dashedrule}{\mbox{%
		\solidrule[2mm]\hspace{2mm}\solidrule[2mm]}}

\DeclareRobustCommand{\tikzcircle}[1]{\tikz{\filldraw[#1] (0,0) circle (0.5ex);}}	
	
	
\DeclareRobustCommand{\squaret}[1]{\tikz{\draw[#1,thick] (0,0) rectangle (0.2cm,0.2cm);}}
\DeclareRobustCommand{\circlet}[1]{\tikz{\draw[#1,thick] (0,0) circle [radius=0.1cm];}}
\DeclareRobustCommand{\trianglet}[1]{\tikz{\draw[#1,thick] (0,0) --
		(0.25cm,0) -- (0.125cm,0.25cm) -- (0,0);}}
\DeclareRobustCommand{\crosst}[1]{\tikz{\draw[#1,thick] (0cm,0cm) --
		(0.1cm,0.1cm) -- (0cm,0.2cm) -- (0.1cm,0.1cm) -- (0.2cm,0.2cm) -- (0.1cm,0.1cm)-- (0.2cm,0cm);}}
\DeclareRobustCommand{\diamondt}[1]{\tikz{\draw[#1,thick] (0,0) --(0.1cm,0.15cm) -- (0.2cm,0cm) -- (0.1cm,-0.15cm) -- (0,0)  ;}}
\DeclareRobustCommand{\squareF}[1]{\tikz{\filldraw[#1,fill opacity= 0.3] (0,0) rectangle (0.2cm,0.2cm);}}

\begin{document}

\maketitle

\vspace{-0.3in}
\noindent
\rule{\linewidth}{0.4pt}

\tableofcontents

%-------------------------------------------------
\section{Equations}
%-------------------------------------------------
The Serre equations with surface tension were derived by \cite{Mitsotakis-etal-2017-1719}
\begin{subequations}
	\begin{align}
	\begin{split}
	\dfrac{\partial h}{\partial t} + \dfrac{\partial (hu)}{\partial x} = 0
	\label{eq:gSGNh}
	\end{split}\\
	\begin{split}
	\dfrac{\partial (hu)}{\partial t} + \dfrac{\partial }{\partial x} \left( hu^2 + \frac{1}{2}gh^2 + \frac{1}{3} h^3 \Gamma  - \tau \mathcal{T}\right)= 0
	\label{eq:gSGNuh}
	\end{split}\\
	\begin{split}
	\dfrac{\partial\left(\mathcal{E}\right)}{\partial t} +\dfrac{\partial}{\partial x}\left[hu \left(\frac{u^2}{2} + \frac{h^2}{6}\frac{\partial u}{\partial x}\frac{\partial u}{\partial x} + gh + \frac{h^2}{3} \Gamma - \tau \frac{\partial^2 h}{\partial x^2} \right) + \tau \frac{\partial h}{\partial x} \frac{\partial }{\partial x}\left(uh\right) \right] = 0
	\label{eq:gSGNE}
	\end{split}
	\end{align}
	where
	\begin{align}
	\Gamma &= \left[\frac{\partial u}{\partial x}\frac{\partial u}{\partial x} - \frac{\partial^2 u}{\partial x \partial t} - u\frac{\partial^2 u}{\partial x^2}\right]\\
	\mathcal{T} &= h \frac{\partial^2 h}{\partial x^2} - \frac{1}{2}\frac{\partial h}{\partial x}\frac{\partial h}{\partial x}\\
	\mathcal{H} &=  \frac{1}{2} \left[uh^2 +  \frac{h^3}{3} \frac{\partial u}{\partial x}\frac{\partial u}{\partial x} + g h^2 + \tau \frac{\partial h}{\partial x}\frac{\partial h}{\partial x} \right]
	\end{align}
	\label{eq:gSGN}
\end{subequations}




Can be rearranged to be
\begin{subequations}
	\begin{align}
	\begin{split}
	\dfrac{\partial h}{\partial t} + \dfrac{\partial (hu)}{\partial x} = 0
	\label{eq:gSGNh_G}
	\end{split}\\                                                                                                                                               
	\begin{split}
	\dfrac{\partial G }{\partial t}  + \dfrac{\partial}{\partial x} \left ( uG + \dfrac{gh^2}{2} - \frac{2}{3} h^3\dfrac{\partial u}{\partial x}\dfrac{\partial u}{\partial x}  - \tau \mathcal{T} \right) = 0
	\label{eq:gSGNuh_G}
	\end{split}\\
	\end{align}
	where
	\begin{align}
	G &= hu - \frac{1}{2}\left(\frac{2}{3} + \beta_1\right) \dfrac{\partial }{\partial x} \left ( h^3 \dfrac{\partial u}{\partial x} \right )
	\end{align}
\end{subequations}

\section{Peakon}
The peakon solution of the serre equations with surface tension are
\begin{subequations}
	\begin{align}
	h(x,0) &= a_0 + a_1\exp\left(-\frac{\sqrt{3}}{a_0} \left|x - ct\right|\right) \\
	u(x,0) &= c \left(\frac{h(x,0) - a_0}{h(x,0)}\right)
	\end{align}
	where
	\begin{align}
	c = \sqrt{1 + \frac{a_1}{a_0}} \\
	g= 1                                                                                                                                                                                                                                                                                                                                                                                                                                                                                                                                                                                                                                                                                                                                                                                                                                                                                                                                                                                                                                                                                                                                                                                                                                                                                                                                                                                                                                                                                                                                                                                                                                                                                                                                                                                                                                                                                                                                                                                                                                                                                                                                                                                                                                                                                                                                                                                                                                                                                                                                                                                                                                                                                                                                                                                                                                                                                                                                                                                                                                                                                                                                                                                                                                                                                                                                                                                                                                                                                                                                                                                                                                                                                                                                                                                                                                                                     
	\end{align}
\end{subequations}

\begin{equation}
G = uh - h^2 h_x u_x - h^3 u_{xx} / 3
\end{equation}
\begin{align}
u_x &= c a_0 \frac{h_x}{h^2}  u_{xx} &= c a_0 \frac{h h_{xx} - 2h_x^2}{h^3} \\
h_x &=   h_{xx} &= 
\end{align}

\section{References}
\bibliographystyle{unsrtnat}
\bibliography{Bibliography}

\end{document} 