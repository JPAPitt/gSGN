\documentclass[10pt]{article}
\usepackage{amsmath}
\begin{document}
Lemma : For $a \ge 0$, $b \ge 0$ and $x \ge 0 $ we have that (all quantities are reals)

\begin{equation}
f(x) = \dfrac{a x^2 + 2}{ b x^2 + 2}
\end{equation}

\begin{equation}
f(x) = 1 + \dfrac{\left(b - a \right) x^2}{ b x^2 + 2}
\end{equation}

\begin{equation}
f(x) = 1 + \dfrac{\left(b - a \right)}{ b +  \dfrac{2}{x^2}}
\end{equation}

is a montone function. We know that $\frac{2}{x^2}$ is monotone decreasing when $x \ge 0$.

Proof:
\[x_0 \le x_1 \implies x_0^2 \le x_1^2 \implies \dfrac{2}{x_1^2} \le \dfrac{2}{x_0^2}   \]

It then follows that $b +  \frac{2}{x^2}$ is monotone decreasing when $x \ge 0$. Then $\dfrac{1}{b +  \frac{2}{x^2}}$ is monotone increasing. Since $x_0 \le x_1 \implies  \dfrac{1}{b +  \frac{2}{x_0^2}} \ge \dfrac{1}{b +  \frac{2}{x_1^2}}$, due to inequality flipping for division. 

Then if $a \le b$ it follows that $f(x)$ is monotone increasing whilst if $a \ge b$ then $f(x)$ is monotone decreasing. With the special case of $a = b$ giving the constant function $f(x) = 1$.


\end{document}