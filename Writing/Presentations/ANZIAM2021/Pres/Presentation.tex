\documentclass[pdf]{beamer}
\mode<presentation>{}
\usetheme{Dresden}
\usepackage{apalike}
\usepackage{graphicx}
\usepackage{movie15}
\usepackage{mwe,tikz}
\usepackage[percent]{overpic}
\beamertemplatenavigationsymbolsempty
%% preamble
\title{Non-linear dispersive water wave models}
\usepackage{subcaption}
\author{Jordan Pitt, Stephen Roberts and Christopher Zoppou \\ Australian National University}
\newcommand\solidrule[1][0.25cm]{\rule[0.5ex]{#1}{1pt}}
\newcommand\dashedrule{\mbox{\solidrule[2mm]\hspace{2mm}\solidrule[2mm]}}
\newcommand{\dotrule}[1]{%
	\parbox[]{#1}{\dotfill}}

\setbeamertemplate{itemize item}[triangle]

\newcommand\T{\rule{0pt}{3ex }}       % Top table strut
\newcommand\B{\rule[-4ex]{0pt}{4ex }} % Bottom table strut

\begin{document}
	
%% title frame
\begin{frame}
\titlepage
\end{frame}
%% normal frame
	

%Do a brief show pictures of water waves, hazards posed
%Move onto the equations, with a picture
%Method: in terms of polynomial representation, FEM, FVM
%Validation: Present results of linear analysis: stability and dispersion analysis, then numerical solution comparisons

\begin{frame}{Outline}
	\begin{itemize}
		\item Motivation
		\item Equations
		\item Linear Theory
		\item Comparison To Numerical Solutions
	\end{itemize}
\end{frame}
\section{Motivation}
%Wave modelling
\begin{frame}{Motivation - Water Waves}
We require accurate models of water waves to understand natural hazards in particular
	\begin{itemize}
		\item Tsunamis
		\item Storm Surges
	\end{itemize}
\end{frame}

\begin{frame}{Motivation}
\begin{figure}
	\centering
	\begin{subfigure}{0.49\textwidth}
		\includegraphics[width=0.9\textwidth]{./Pics/Web/SualwesiTsunami.jpg}
		\caption{Sulawesi Tsunami (Indonesia, 2018).}
	\end{subfigure}
	\begin{subfigure}{0.49\textwidth}
	\includegraphics[width=1.1\textwidth]{./Pics/Web/HurricaneFlorence.jpg}
	\subcaption{Hurricane Florence (U.S.A, 2018)}
	\end{subfigure}
\end{figure}
\end{frame}


\begin{frame}{Motivation - Water Waves}
We require accurate models of water waves to understand natural hazards in particular
\begin{itemize}
	\item Tsunamis
	\item Storm Surges
\end{itemize}
\bigskip
Current models built on non-dispersive models where wave speed independent of frequency (Shallow Water Wave Equations). 
\pause

\bigskip
\textbf{{What's the effect of dispersion on these natural hazards?}}
\end{frame}

\begin{frame}{Today's Focus}
\begin{itemize}
	\item Compare linear theory and numerical solutions for the models of interest.
\end{itemize}
\bigskip
\pause
\emph{Which family of equations?}
\end{frame}


\section{Family of Equations}
\begin{frame}{Depth Averaged Set Up}
%Equations for conservation of mass and momentum written in terms of the water depth $h(x,t)$, the depth average horizontal velocity $u(x,t)$ and acceleration due to gravity $g$.
\begin{figure}
	\centering
	\includegraphics[width=0.9\textwidth]{./Pics/Tex/Explanatory/Setupplot/Waves.pdf}
	\caption{Relevant Quantities.}
\end{figure}
\end{frame}


\begin{frame}{Generalised Serre-Green-Naghdi Equations}
%conservation of mass (exact)
%conservation of momentum (approximation)
%Our h, u and g from before, and also now some free paramters - beta1 and beta2
%Equations also possess an equation for conservation of energy (which holds when quantities are sufficiently smooth)
%when beta's are 0, we get the familar non-dispersive SWWE
%For a physical interpretation, the betas will determine our approximation to the vertical velocity of the water at the free surface, when betas are 0, the vertical velocity is 0, otherwise it is some value dependent on derivatives of h and u in the latter case it is linear throghout depth
\begin{align*}
&\dfrac{\partial h}{\partial t} + \dfrac{\partial (hu)}{\partial x} = 0\\
&\dfrac{\partial (hu)}{\partial t} + \dfrac{\partial }{\partial x} \left( hu^2 + \frac{1}{2}gh^2  +  {\color{blue} {\beta_1} \Phi } -   {\color{red} {\beta_2}\Psi}  \right)= 0 \\
\end{align*}
where
\begin{align*}
\color{blue} \Phi  & \color{blue}= \frac{h^3}{2}\left( \frac{\partial u}{\partial x}\frac{\partial u}{\partial x} - \frac{\partial^2 u}{\partial x \partial t} - u\frac{\partial^2 u}{\partial x^2}\right) \\
\color{red} \Psi & \color{red}=  \frac{gh^2}{2} \left(h \frac{\partial^2 h}{\partial x^2} + \frac{1}{2} \frac{\partial h}{\partial x}\frac{\partial h}{\partial x}\right)
\end{align*}
\end{frame}

\begin{frame}{Generalised Serre-Green-Naghdi Equations - Properties}
\begin{itemize}
	\item Conserves mass, momentum and energy
	\item Reduces to well known equations for particular $\beta$ values
	\item Depth averaged equations have been very successful for large scale models
	\item Nice linear dispersion properties as well
\end{itemize}
\end{frame}


\begin{frame}{Linear Dispersion Relationship}
\begin{itemize}
	\item Linearise equations - considering waves of small amplitude 
	\item Relate angular frequency ($\omega$) to  wave number ($k$)
\end{itemize}
\end{frame}

\section{Linear Theory}    
    
%Considering wave properties of small amplitude waves on a large background of still water
\begin{frame}{Linearise equations (Space)}
\begin{figure}
	\centering
	\includegraphics[width=0.9\textwidth]{./Pics/Tex/Explanatory/DispersionPlot/Dispersion.pdf}
\end{figure}
\end{frame}

\begin{frame}{Linearise equations (Space/ {\color{red} Time})}
\begin{figure}
	\centering
	\includegraphics[width=0.9\textwidth]{./Pics/Tex/Explanatory/DispersionPlot/Dispersion_Analagous.pdf}
\end{figure}
\end{frame}

\begin{frame}{Dispersion Relations of generalised Serre-Green-Naghdi and Water (linearised)}
\begin{align*}
\omega_{\text{gSGN}} & =  k \sqrt{gH} \sqrt{\dfrac{{\color{red}\beta_2 H^2 k^2} + 2}{ {\color{blue}\beta_1 H^2 k^2} + 2}} \\
\omega_{\text{water}} & = \sqrt{gk \tanh\left(kH\right)}
\end{align*}
\end{frame}


\begin{frame}{Taylor Series Expansions}

{\footnotesize
\begin{align*}
\omega_{\text{gSGN}} & = \sqrt{gH} k &&- \sqrt{\frac{ {\color{blue}\beta_1} - {\color{red}\beta_2}  }{2}} \sqrt{gH} H  k^2 &&+  \sqrt{\frac{{\color{blue}\beta_1}\left({\color{blue}\beta_1}- {\color{red}\beta_2}\right) }{4}}  \sqrt{gH} H^2 k^3 &&+ \mathcal{O}\left(k^4\right)\\
\omega_{\text{water}} & = \sqrt{gH} k &&- \sqrt{\frac{1}{3}} \sqrt{gH} H k^2   &&+  \sqrt{\frac{2}{15}}\sqrt{gH} H^2 k^3  &&+ \mathcal{O}\left(k^4\right) 
\end{align*}}
\end{frame}


\begin{frame}{Accuracy Summary Plot}
\centering
	\includegraphics[width=0.75\textwidth]{./Pics/Tex/Explanatory/RegionsPlot/AccuracySummary.pdf}
\end{frame}


\begin{frame}{Phase Speed ($c$)}
Looking at the phase speed  $c_{\text{gSGN}} = \dfrac{\omega_{\text{gSGN}}}{k}$ we get
\begin{align*}
c_{\text{gSGN}} &= \sqrt{gH} \sqrt{ \dfrac{{\color{red}\beta_2 H^2 k^2} + 2}{ {\color{blue}\beta_1 H^2 k^2} + 2}}
\end{align*}
%If we can bound the phase speed as a function of $k$ we can understand the maximum and minimum wave speeds and determine the `width' and direction of dispersive wave trains. 
\end{frame}

\begin{frame}{Phase Speed ($c$) Bounds}
The phase speed can be bounded in the following way with three cases %(bounding the speed of waves, and giving the region in wh)
\begin{itemize}
	\item When ${\color{blue}\beta_1 } > {\color{red}\beta_2 }$ we have
	\begin{align*}
0 \le \sqrt{\frac{{\color{red}\beta_2 }}{{\color{blue}\beta_1 }}} \sqrt{gH} \le  c_{\text{gSGN}} \le  \sqrt{gH}
	\end{align*} 
	\item When ${\color{blue}\beta_1 } = {\color{red}\beta_2 }$ we have
	\[c_{\text{gSGN}} = \sqrt{gH}\]
	\item When ${\color{blue}\beta_1 } < {\color{red}\beta_2 }$ we have
	\[ 0 \le  \sqrt{gH} \le c_{\text{gSGN}} \le \sqrt{\frac{{\color{red}\beta_2 }}{{\color{blue}\beta_1 }}} \sqrt{gH}  \]
\end{itemize}

\end{frame}


\begin{frame}{Phase Speed Regions }
Because $\sqrt{gH}$ controls the speed of shocks these chains of inequalities lead to the following behaviours:
\begin{itemize}
	\item When ${\color{blue}\beta_1 } > {\color{red}\beta_2 }$ we have dispersive waves form behind shocks
	\[c_{\text{gSGN}} \le  \sqrt{gH}\]
	\item When ${\color{blue}\beta_1 } = {\color{red}\beta_2 }$ we have no dispersive waves
	\[c_{\text{gSGN}} =  \sqrt{gH}\]
	\item When ${\color{blue}\beta_1 } < {\color{red}\beta_2 }$ we have dispersive waves form ahead of shocks 
	\[c_{\text{gSGN}} \ge  \sqrt{gH}\]
\end{itemize}
\end{frame}
\begin{frame}{Accuracy Summary Plot}
\centering
\includegraphics[width=0.75\textwidth]{./Pics/Tex/Explanatory/RegionsPlot/AccuracySummaryWithRegions.pdf}
\end{frame}

\section{Comparison To Numerical Solutions}
\begin{frame}{Comparsion Between Linear Theory and Numerical Solutions}
Are these widths replicated in the numerical solutions of the non-linear dispersive wave equations? \newline \newline
We answer this by comparing to numerical solutions to the dam-break initial condition problem. 
\end{frame}

\begin{frame}{Dambreak Problem Initial Conditions}
\centering
\includegraphics[width=0.73\textwidth]{./Pics/Tex/Results/DB/Init/Init.pdf}
\end{frame}

\begin{frame}{Shallow Water Wave Equations \hfill \includegraphics[width=0.17\textwidth]{./Pics/Tex/Explanatory/RegionsPlot/SPSWWE.pdf}}
\centering
\includegraphics[width=0.7\textwidth]{./Pics/Tex/Results/DB/SWWE/SWWE.pdf}
\end{frame}


\begin{frame}{Serre Equations \hfill \includegraphics[width=0.17\textwidth]{./Pics/Tex/Explanatory/RegionsPlot/SPSerre.pdf}}
\centering
\includegraphics[width=0.7\textwidth]{./Pics/Tex/Results/DB/Serre/Serre.pdf}
\end{frame}

\begin{frame}{Serre Equations \hfill \includegraphics[width=0.17\textwidth]{./Pics/Tex/Explanatory/RegionsPlot/SPSerre.pdf}}
\centering
\includegraphics[width=0.7\textwidth]{./Pics/Tex/Results/DB/Serre/RegionSerre.pdf}
\end{frame}

\begin{frame}{Optimal Dispersion Equations \hfill \includegraphics[width=0.17\textwidth]{./Pics/Tex/Explanatory/RegionsPlot/SPiSGN.pdf}}
\centering
\includegraphics[width=0.62\textwidth]{./Pics/Tex/Results/DB/iSGN/RegioniSGN.pdf}
\end{frame}


\begin{frame}{Conclusion}
\centering
\includegraphics[width=0.99\textwidth]{./Pics/Tex/Explanatory/RegDBV/3x3GridNoMovement.pdf}
\end{frame}


\begin{frame}{Changing $\beta$ values}
Now that we have gone through in depth each of the particular $\beta$ values, we can begin to understand what happens if we change $\beta$ values over time. Coming back to the original video at the beginning of the talk.
\end{frame}

\begin{frame}{Changing $\beta$ values Outline}
\centering
\includegraphics[width=0.99\textwidth]{./Pics/Tex/Explanatory/RegDBV/3x3Grid.pdf}
\end{frame}

\begin{frame}[plain]{}
\begin{tikzpicture}[remember picture,overlay]
\node[anchor=north east, inner sep=0pt] at (current page.north east) {
\includemovie[
poster,
text={}
]{\paperwidth}{\paperheight}{./Videos/Dambreak.avi}
};
\end{tikzpicture}
\end{frame}

\begin{frame}{Expanded Grid}
\centering
\includegraphics[width=0.7\textwidth]{3x3GridAhead.pdf}
\end{frame}

\begin{frame}{Ahead Of Shock}
\centering
\includegraphics[width=0.7\textwidth]{hEx04.pdf}
\end{frame}

\end{document}